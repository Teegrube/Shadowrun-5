\section{Magie}

\subsection{Hausregeln}
\begin{itemize}
    \item Beim Entzugswiderstand werden immer nur die natürlichen Attribute des widerstehenden Charakters verwendet. Technologische und/oder magische Attributsverstärkungen werden dabei also nicht angerechnet.
    \item Reagenzien können nicht verwendet werden, um Limits zu hacken.
    \item Ein Charakter kann von jeder Fokusart immer nur einen gleichzeitig aktiv haben. Beispielsweise also einen Zauberspeicher (insgesamt; nicht pro Zauberart), einen Herbeirufen-Fokus, einen Metamagie-Fokus usw.
    \item Bei Proben sind Würfelpool-Boni durch Foki auf $4$ begrenzt.
    \item Foki haben andere Verfügbarkeiten. Kraftfoki haben $6\text{E}/\text{Kraftstufe}$, Waffenfoki haben $5\text{E}/\text{Kraftstufe}$, alle anderen haben $4\text{E}/\text{Kraftstufe}$.
    \item Die maximale Kraftstufe herbeigerufener Geister ist gleich dem $\text{Magieattribut}$ des Beschwörers.
    \item Ein Charakter kann - wie gehabt - maximal einen ungebundenen Geist draußen haben. Für gebundene Geister gilt, dass das Gesamtmaximum an beschworenen Geistern $2$ ist.
    \item Beim Verstärkten Panzer herbeigerufener Geister fällt der Schwellenwert weg.
    \item Beschwörer können mittels des Systems für Astrale Reputation (in \autoref{sec:Astrale Reputation}) weitere Effekte für ihre beschworenen Geister erzielen.
\end{itemize}

\subsection{Erweiterte magische Fertigkeiten}
Die Fertigkeiten \textbf{Fokusherstellung} und \textbf{Entzaubern} haben zusätzliche Anwendungsgebiete. Genauer gesagt werden im Verlauf der Kampagne verzauberte Objekte auftauchen, welche keine Foki sind. Um mit diesen zu interagieren, werden diese beiden Fertigkeiten benötigt.

\subsection{Arten der Magie}
\paragraph{Magieradepten}
Für Magieradepten kosten Kraftpunkte je $10\text{Karma}$, statt wie RAW 5.\\

\subsubsection{Aspektzauberer}
Aspektzauberer haben zwar den Nachteil der Eingeschränktheit in der Wahl ihrer Mittel. Dafür bekommen sie hier jedoch einige Boni - je nach Spezialgebiet. Ein Aspektzauberer erhält bei Erwerb einer Spezialisierung/Expertise auf eine dieser Fertigkeiten den entsprechenden Bonus.
\paragraph{Aspektbeschwörer}
\begin{itemize}
    \item[Berschwören] Der Charakter hat einen zusätzlichen Herbeirufen-Würfel.
    \item[Binden] Der Charakter hat einen zusätzlichen Binden-Würfel.
    \item[Verbannen] ?
\end{itemize}

\paragraph{Aspektverzauberer}
\begin{itemize}
    \item[Alchemie] Alchemische Erzeugnisse haben eine um $1$ höhere Wirksamkeit.
    \item[Fokusherstellung] ?
    \item[Entzaubern] ?
\end{itemize}

\paragraph{Aspekthexer}
\begin{itemize}
    \item[Spruchzauberei] Der Charakter kann einen zusätzlichen Zauber ohne Würfelpoolmalus aufrechterhalten. 
    \item[Ritualzauberei] Der Charakter
    \item[Antimagie] Der Charakter
\end{itemize}

\subsection{Magische Vor-/Nachteile}
\paragraph{Astralsucht}
Ein Charakter mit diesem Vor-/Nachteil kann astral wahrnehmen, OHNE den Würfelpoolmalus von $-2$ auf alle physischen Proben zu erhalten. Dafür erhält er einen Malus von $-2$ auf all seine Proben, wenn er NICHT astral wahrnimmt.


\subsection{Geister}
\subsubsection{Astrale Reputation}
Die astrale Reputation eines Charakters kann einen Wert von $-10$ bis $10$ annehmen. Diese hat folgende Auswirkungen:
\begin{itemize}
    \item $\geq +9$: 3 Herbeirufen-Würfel
    \item $\geq +6$: 2 Herbeirufen-Würfel.
    \item $\geq +3$: 1 Herbeirufen-Würfel.
    \item $\geq -3$: 2 Binden-Würfel. Der Geist greift einen einmal an, sollte er sich befreien können.
    \item $\geq -6$: 3 Binden-Würfel. Der Geist greift einen 2-3 mal an, sollte er sich befreien können. 
    \item $\geq -9$: 4 Binden-Würfel. Der Geist versucht einen zu töten, sollte er sich befreien können. 
\end{itemize}
\subsubsection{Herbeirufen}
Ruft ein Charakter einen Geist herbei, so werden je nach astraler Reputation Herbeirufen-Würfel geworfen. Dieser hat - je nach Würfelergebnissen - folgende Auswirkungen:
\begin{enumerate}
    \setcounter{enumi}{3}
    \item[1,1] $\text{Entzug}-1$
    \item[3,4] $\text{Dienste}+1$
    \item[5,6] $\text{Kraftstufe}+1$
\end{enumerate}
\subsubsection{Binden}
Bindet ein Charakter einen Geist, wirft die Spielleitung verdeckt einen Binden-Würfel ($1\text{d6}$) für diese Probe. Je nach astraler Reputation können weitere solcher Würfel hinzukommen.\\
Zeigt mindestens die Hälfte der Binden-Würfel eine $1$, befreit sich der Geist. Je nach astraler Reputation kann der Geist dann auch auf seinen Beschwörer losgehen. Bei einem Erfolg hingegen erhöht sich die Kraftstufe nach dem Ende des Binde-Prozesses um $1$.\\
Boni aus dem Herbeirufen-Prozess gehen verloren.
