\section{Magie}
\label{sec:Magie}
SR5 ist bekannt dafür, "Magicrun" zu sein, denn Magie ist hier mit großem Abstand am mächtigsten. Daher ist Magie hier generft.\\
Dies gescheht insbesondere durch folgende Punkte:
\begin{enumerate}
    \item Die Kraftstufe von Geistern ist maximal gleich dem Magieattribut des Zauberers, und ihre verstärkte Panzerung verliert den Mindestschaden.
    \item Einige Fokusarten sind raus und der maximale Würfelpoolbonus durch Foki bei einer Probe ist $+4$, genau wie es auch bei allen mundanen Verbesserungen der Fall ist.
    \item Reagenzien können nicht länger zum Hacken von Entzugswerten verwendet werden.
    \item Der Würfelpoolmalus beim Aufrechterhalten von Zaubern fällt weg.
    \item Verbesserungen von geistigen Attributen durch Magie zählt nicht beim Entzugswiderstand.
\end{enumerate}

Zudem sind Geister in Anlehnung an das astrale Reputationssystem aus Shadowrun 6 einem selbsterstellten Reputationssystem unterworfen.
\subsection{Grundlagen der Magie}
\paragraph{Würfelpools}
Der Würfelpool beim Wirken von Magie ist immer $\text{Magie}+\text{Fertigkeit}$.
\paragraph{HB: Entzug}
\textbf{Entzug} ist der Preis, den man für das Wirken von Magie zahlen muss. Bei jedem Wirken von Magie nimmt ein Charakter Schaden in Höhe des Entzugs der gewirkten Magie. 
\homebrew{Reagenzien können nicht zur Minderung des Entzugs verwendet werden. Magisch oder technologisch erhöhte Attribute verbessern nicht den Entzugswiderstandspool.}

\subsection{Wahrnehmen von Magie}
Plötzliche, unplausible magische Erscheinungen erkennen Charaktere jedoch direkt als magische Fälschungen. Taucht beispielsweise plötzlich vor den Augen eines Charakters aus dem Nichts eine Armee von Feinden auf, wird diesem ziemlich schnell klar sein, dass es sich um ein Trugbild handelt. Dennoch kann ihn diese Erscheinung - falls er dem Zauber nicht widerstanden hat - im ersten Moment aus der Fassung bringen.\\
Auch erfordert jeder Zauber immer eine Geste und einen Zauberspruch. Je höher die Kraftstufe des Zaubers, desto auffälliger dessen Wirken. Es gelten dafür die Regeln zum Wahrnehmen von Magie:\\
Dazu legt man eine Probe auf $\text{Wahrnehmung} + \text{Intuition} \left[Geistig\right]$ gegen $\text{Fertigkeitsstufe} - \text{Kraftstufe}$ des Zaubernden/des Zaubers ab, wobei der Schwellenwert immer mindestens 1 beträgt. \\
\homebrew{Außerdem merkt man Charakteren, welche Zauber aufrechterhalten, dies mitunter auch an. Das Aufrechterhalten von Zaubern erfordert ein großes Maß an Konzentration und Anstrengung. Zaubernde beginnen dadurch beispielsweise, zu Schwitzen, wirken abgelenkt, etc. Zwar wird natürlich nicht direkt ersichtlich, dass dies durch das Wirken von Magie bedingt ist, aber manchmal gibt es weitere Hinweise, welche ein vernunftbegabter Charakter gegebenenfalls auf die Idee bringt, es könnte sich um Magie handeln.}

\subsection{Reagenzien}
Reagenzien können nicht dazu verwendet werden, Limits zu ändern.

\subsection{Spruchzauberei}
\subsubsection{Gedankenmagie}
Diese Magie umfasst alle Manipulationszauber der Kategorie \textbf{Beherrschung}, den Wahrnehmungszauber \textbf{Geistessonde} und alle anderen Zauber, welche in die mentale Autonomie ihres Ziels eindringen. Da dies ein extremer Eingriff in die Autonomie des Ziels darstellt, werde ich dieses Thema niemals auf die leichte Schulter nehmen. Derartige Magie gehört daher in meinen Runden zu den \textbf{Verbotenen Künsten} (genau wie toxische Magie, Blutmagie und Insektenmagie) und erfordert die neu eingeführte Metamagie \textbf{Gedankenmagie}, um sie zu erlernen.

\subsubsection{Heilungszauber}
\paragraph{HB: Gesteigerte Reflexe}
\homebrew{Die Steigerungen durch diesen Zauber ist nicht kumulativ mit anderen Initiativeerhöhungen.}
\textit{Dieser Zauber wäre sonst die einzige Ausnahme von dieser Regel.}

\subsubsection{Illusionsmagie}
Als Klarstellung: Wie im GRW auf S.283 beschrieben wirken Illusions-Manazauber auf das Bewusstsein des Ziels. Diese Zauber gaukeln dem \textit{Verstand} des Ziels vor, bspw etwas zu sehen, was gar nicht da ist. Da dies nicht die Sicht, sondern der Verstand direkt betroffen ist, hilft es nicht, sich einen Kamerafeed des Gesehenen anzuschauen, um den Zauber zu umgehen. Dafür werden die Leute im Kameraraum die Illusion gar nicht wahrnehmen können. Jede*r, der/die diesem Zauber widerstanden hat, sieht gar nichts, kann aber mit einer entsprechenden Probe zum Wahrnehmen von Magie (S.278 GRW) erkennen, dass sie/er gerade einem Zauber widerstanden hat.\\
Physische Illusionszauber hingegen erschaffen bspw eine tatsächliche Lichterscheinung. Diese werden daher auch von Kameras wahrgenommen, welche jedoch oft bessere Widerstandspools gegen den Zauber haben, als NPCs. Bei erfolgreichem Widerstand gegen den Zauber erkennt ein Charakter oder eine Kamera Imperfektionen im dargestellten Bild und dieses daher als Fälschung. Das magische Bild sehen aber dennoch alle.

\paragraph{(Trideo-) Trugbild}
Zur Klarstellung: Werden Trugbilder gewirkt, so legt ein Charakter vorher fest, was passiert. Danach kann er daran nix mehr ändern. Siehe auch hier https://www.reddit.com/r/Shadowrun/comments/4y1vc9/5e\_tridphantasm\_power\_and\_limitations/, falls Fragen aufkommen.\\
Außerdem: Illusionen, welche offensichtlich Quatsch sind, werden sofort als solche durchschaut werden.

\subsection{Foki}
Kraftfoki und Zauberspeicher sind raus. Die maximalen Würfelpoolboni von Foki zu einer Probe dürfen - wie auch bei allen anderen Proben - 4 nicht überschreiten.

\subsection{HB: Initiation}
\label{sec:Initiation}
Die Initiation funktioniert regeltechnisch größtenteils wie RAW, jedoch mit einigen Modifikationen: \homebrew{Der maximale Initiatengrad ist begrenzt, und zwar auf die (abgerundete) Hälfte des Magieattributs eines Charakters.}\\
\homebrew{Bereits bei der Charaktererschaffung kann mit Karma bis zu ein Initiatengrad erworben werden. Dafür ist der maximale bei der Charaktererschaffung mögliche Wert des Magieattributs auf $6$ begrenzt.}
\subsubsection{Metamagie}
Zentrierung ist raus.


\subsection{Adept}
\paragraph{HB: Vorteil: Weg des Ausgebrannten}
\label{qual:Ausgebrannt}
Der Vorteil \textbf{Weg des Ausgebrannten} (SG S.207f) hat nun folgende Wirkung:\\
\homebrew{15 Karma:\\
Ein Adept, welcher durch Bodytech mehr als 2 Punkte Essenz verloren hat, kann diesen Weg beschreiten. Dies erlaubt ihm, bei seiner Initiation \ref{sec:Initiation} stattdessen den Effekt einer Cyberinitiation \ref{sec:Cyberinitiation} zu erhalten. Der Adept erhält jedoch nur ein Essenzloch der Größe $0.5 \text{Essenz}$ statt der für Mundane üblichen $1.0 \text{Essenz}$. Für seinen maximalen Cyberinitiatengrad gilt jedoch die selbe Einschränkung wie bei der normalen Cyberinitiation: Der maximale Cyberinitiatengrad ist immer gleich dem (aufgerundeten) halben Essenzverlust eines Charakters. \textbf{Wichtiger Hinweis}: Wählt ein Charakter bei seiner Initiation diese Option, steigen sowohl sein Initiatengrad, als auch sein Cyberinitiatengrad um 1.}

\subsection{Magieradepten}
\homebrew{Magieradepten funktionieren hier wie in Shadowrun 4. Sie haben zwei separate Magieattribute: Eines für die Adeptenseite und eines für die Zaubererseite. Wann immer ein Magieradept sein Magieattribut erhöht, muss er sich entscheiden, ob er diesen zusätzlichen Punkt Magie in die eine oder die andere Seite investiert. Die einzelnen Seiten funktionieren dann jedoch wie ein reiner Adept bzw wie ein reiner Zauberer. Ein Magieradept mit Adept 2 und Zauberer 4 erhält bspw 2KP von der Adeptenseite, während abb einer Kraftstufe von 5 der Entzug seiner Zauber physisch wird.}\\
\homebrew{Außerdem kann ein Magieradept keine Vorteile für Adepten/Zauberer erwerben.}

\section{Liste gebannter Zauber}
Alle "Beherrschung"-Zauber, Geistessonde (und ähnliche), Verbesserte Unsichtbarkeit, Physische Maske.


\subsection{Geister}
Die maximale Kraftstufe von Geistern ist auf das Magieattribut des Beschwörers beschränkt. Außerdem entfällt bei ihrer gehärteten Panzerung der Schwellenwert, über welchen man kommen muss, um sie mit mundanen Waffen zu verletzen.\\
Man kann zu jeder Zeit immer nur einen aktiven Geist im Spiel haben - auch, wenn dieser gebunden ist.\\
Man kann auch zu jeder Zeit nur maximal einen Geist einer Geisterart gebunden haben und auch keinen weiteren Geist dieser Art normal herbeirufen, bis die Bindung aufgehoben ist.  
\subsubsection{Astrale Reputation}
Ein Charakter genießt bei jeder Geisterart einen astralen Ruf von $-10$ bis $+10$. Charaktere verteilen bei ihrer Erschaffung gemäß ihrer Tradition unterschiedliche Mengen an Punkten von astralem Ruf auf die verschiedenen Geisterarten. Je nach Ruf bei einer Geisterart, hat ein Charakter ein anderes Verhältnis zu dieser.\\
Man kann maximal mit $\pm 4$ Geisterruf bei einere Geisterart ins Spiel starten.\\
Charismabasierte Traditionen verteilen $+10$ und $-5$ Geisterruf. Diese Traditionen sehen Geister als Partner auf Augenhöhe an und legen bspw Wert darauf, einen Geist nach seinem Namen zu fragen.\\
Intuitionsbasierte Traditionen verteilen $+7$ und $-7$ Geisterruf. Diese Traditionen sehen Geister in etwa wie andere Geschäftspartner betrachten: Am Verhandlungstisch sind sie Gegner, welche es auszustechen gibt, doch um der Beziehungen Willen, lädt man sie nach der Verhandlung auch gerne mal zum Essen ein.\\
Logikbasierte Traditionen verteilen $+5$ und $-10$ Geisterruf. Diese Traditionen halten Geister für unter ihnen stehende Werkzeuge. Sie haben keine Hemmungen, einen Geist zu binden, zu verdrängen oder erneut zu rufen. Diese fragen einen Geist beispielsweise nie nach seinem Namen.\\
\\
Aktionen, welche für Geister unangenehm sind (wie das Binden, das Betrauen mit Aufgaben, welche nicht zu dem bei der Traditionsbeschreibung genannten Aufgabenfeld oder das erneute Herbeirufen vor dem Tag-/Nacht-Wechsel) verringern den Geisterruf. Respektvoller Umgang mit Geistern, wie beispielsweise das Erfüllen von Missionen für den Geist oder das Fragen nach seinem Namen erhöhen den Geisterruf.\\
\\
Ein negativer Geisterruf hat den Nachteil, dass Geiuster einen beispielsweise bei Bewusstseinsverlust angreifen können. Dafür hat es den Vorteil, dass man sie rücksichtslos binden und damit dem Entzug \textit{während} einer Mission aus dem Weg gehen, sowie alle sonstigen Vorteile des Bindens genießen kann.\\
Ein positiver Geisterruf hat den Nachteil, dass man eingeschränkt dabei ist, Geister zu verwenden, wie man will. Außerdem vershlechtert sich der Ruf beim Binden, sodass dieses nur sehr begrenzt eingesetzt werden kann. Der Vorteil ist dafür, dass Geister sehr freundlich zu einem sind. Außerdem werden sie bei einem guten Ruf oft mit einer höheren Kraftstufe erscheinen, einen zusätzlichen Dienst anbieten, oder ihren Entzug verringern.\\
\\
Konkret ist es so:
\begin{itemize}
    \item $\geq +9$: Wirf 1d6. 1: +1 KS. 2: +1 Dienst. 3: -1 Entzug. 4: -2 Entzug. 5: +2 Dienste. 6: +2 KS.
    \item $\geq +6$: Wirf 1d6. 1: +1 KS. 2: +1 Dienst. 3: -1 Entzug. 4: -1 Entzug. 5: +1 Dienste. 6: +1 KS.
    \item $\geq +3$: Wirf 1d6. 1: +1 KS. 2: +1 Dienst. 3: -1 Entzug. Sonst: Kein Effekt.
    \item $\geq -3$: Der Geist greift einen einmal an, sollte er sich befreien können.
    \item $\geq -6$: Der Geist greift einen 2-3 mal an, sollte er sich befreien können. 
    \item $\geq -9$: Der Geist versucht, einen zu töten, sollte er sich befreien können. 
\end{itemize}
Die eventuellen Effekte eines positiven Geisterrufs verschwinden beim Binden.\\
\\
\subsubsection{Der Umgang mit Geistern}
Geister werden als eine Art Connection behandelt. Je nachdem, welcher Tradition man anhängt, funktioniert dies anders.\\
Bei charismabasierten Traditionen ist es üblich, pro Geisterart immer die gleiche Entität zu rufen. Diese wird oft bei der Ausbildung durch Traumreisen, astrale Questen oder Ähnliches gesucht, wobei man sich mit ihr anfreundet. Manchmal hat man auch zwei oder drei Geisterkumpels einer Art - jeweils mit einer eigenen Persönlichkeit.\\
Bei intuitionsbasierten Traditionen hat man durchaus Geisterkumpels (rein geschäftlich, natürlich!), aber man ruft für irgendwelche random Aufgaben auch gerne mal random Geister herbei.\\
Bei logikbasierten Traditionen ist es den meisten Leuten egal, welchen Geist sie rufen. Doch auch hier empfinden einige Vergnügen daran, immer die gleiche EEntität wieder und wieder zu knechten und sich dauerhaft gefügig zu machen.









\subsubsection{Spruchzauberei}
Ablauf des Zauberwirkens:
\paragraph{1. Auswahl des Zaubers}
Wähle einen oder mehrere (bis zu $\text{Magie}$) Zauber aus, die dein Charakter kennt und teile deinen Würfelpool unter ihnen auf. Modifikationen durch Spezialisierungen, Foki, etc. werden erst nach dem Aufteilen angerechnet.
\paragraph{2. Auswahl des Ziels}
Für Flächenzauber (Zaubern deren Reichweite mit (F) gekennzeichnet ist), wird ein beliebiger Punkt im Raum (in der Regel im Blickfeld des Zaubernden) gewählt.\\
Für Zauber ohne (F) wird ein einzelnes - je nach Zauberbeschreibung und Reichweite legales Ziel gewählt.\\
\textit{Mehr dazu unter \ref{mag:Reichweite}}
\paragraph{3. Auswahl der Kraftstufe}
Die Kraftstufe bildet das Limit der Spruchzaubereiprobe. Bei physischen Kampfzaubern hängt zusätzlich noch der Schadenscode von der Kraftstufe ab.
\paragraph{4. Zauber wirken}
Das Wirken eines Zaubers erfordert eine \textbf{Komplexe Handlung}. Der Charakter legt eine Probe auf $\text{Magie} + \text{Spruchzauberei}$ ab und berücksichtigt dabei Modifikatoren durch Verletzungen, aufrechterhaltene Zauber, etc. Je nach Art des Zaubers wird ihm unterschiedlich widerstanden.
\textit{Schnellzaubern:} Ein Charakter kann diese Option nutzen, um einen Zauber mit einer Einfachen Handlung statt einer Komplexen handlung zu wirken. Dadurch wird der Entzug des Zaubers jedoch um $+3$ erhöht.\\
\begin{table}[h!]
\centering
    \begin{tabular}{c|c c}
        Typ & Art & Widerstandspool \\
        \hline
        Heilzauber   & Essenz         & keine, aber Würfelpoolmalus von $-\text{fehlende Esssenz des Ziels}$ \\
                     & Negativ        & Widerstand mit $\text{Konstitution}$ (Shadow Spells, page 18)\\
        \hline
        Illusion     & Mana           & Lebend: $\text{Willenskraft} + \text{Logik}$; Objekt: immun \\
                     & Physisch       & Lebend: $\text{Willenskraft} + \text{Logik}$; Objekt: $\text{Objektwiderstand}$ \\
        \hline
        Kampf        & Direkt         & Physisch: $\text{Konstitution}$; Mana: $\text{Willenskraft}$ \\
                     & Indirekt       & Einzelziel: Wie Fernkampfangriff; Fläche: Wie Granate (GRW S.182) \\
        \hline
        Manipulation & Umgebung       & Hat kein eigentliches Ziel \\
                     & Transformation & Lebend: $\text{Konstitution} + \text{Stärke}$; Objekt: $\textit{Objektwiderstand}$ \\
        \hline
        Wahrnehmung  & Aktiv          & Lebend: $\text{Willenskraft} + \text{Logik}$; Objekt: $\textit{Objektwiderstand}$; \\
                     &                & Magisch: $\text{Kraftstufe} \times 2$
    \end{tabular}
\end{table}



\subsubsection{Eigenschaften von Zaubersprüchen}

\paragraph{Reichweite}
\label{mag:Reichweite}



\subsection{Würfelpools Zauberer/Ziel}
Zauberer: Magie + Spruchzauberei (+eventuelle Spezialisierung)\\
Verteidiger (Je nach Zauberart)\\
\quad adf


