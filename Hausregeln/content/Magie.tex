\section{Magie}
\label{sec:Magie}
SR5 ist bekannt dafür, "Magicrun" zu sein, denn Magie ist hier mit großem Abstand am mächtigsten. Daher sind hier einige Hausregeln eingearbeitet, welche an Punkten, an denen ein einzelner Zauber oft eine ganze Fähigkeit obsolet macht, generft worden. 
\subsection{Grundlagen der Magie}
\paragraph{HB: Entzug}
\textbf{Entzug} ist der Preis, den man für das Wirken von Magie zahlen muss. Bei jedem Wirken von Magie nimmt ein Charakter Schaden in Höhe des Entzugs der gewirkten Magie. 
\homebrew{Ist die Kraftstufe eines Zaubers kleiner oder gleich dem Magieattribut des Charakters, verursacht der Entzug geistigen Schaden. Ist die Kraftstufe größer, verursacht er stattdessen physischen Schaden.}

\subsubsection{HB: Physische Illusionsauber}
\homebrew{
    Alle RAW physischen Illusionszauber sind gebannt.    
}
\paragraph{HB: Chamäleon}
\homebrew{
    Art: P, Reichweite: Bf, Dauer: A, Entzug: KS-2\\
    Diese Zauber sollen ein Ersatz für \textit{Verbesserte Unsichtbarkeit} sein.\\
    Solange ein Charakter sich nur gehend bewegt (siehe \ref{sec:Bewegung}), erhalten alle, die versuchen, den Charakter per Sicht wahrzunehmen, einen Würfelpoolmalus in Höhe der Erfolge bei der Spruchzaubereiprobe.
}

\subsection{HB: Foki}
\homebrew{Die Verfügbarkeit von Foki ist immer mindestens 13. Dadurch sind sie bei der Charaktererschaffung nicht erhältlich. Insgesamt sind Foki viel seltener, als nach den RAW und ein Charakter kann sich glücklich schätzen, einen einzigen Fokus zu besitzen.}

\subsection{HB: Initiation}
\label{sec:Initiation}
Die Initiation funktioniert regeltechnisch größtenteils wie RAW, jedoch mit einigen Modifikationen: \homebrew{Der maximale Initiatengrad ist begrenzt, und zwar auf die (abgerundete) Hälfte des Magieattributs eines Charakters.}


\subsection{Adept}
\paragraph{HB: Vorteil: Weg des Ausgebrannten}
\label{qual:Ausgebrannt}
Der Vorteil \textbf{Weg des Ausgebrannten} (SG S.207f) hat nun folgende Wirkung:\\
\homebrew{15 Karma:\\
Ein Adept, welcher durch Bodytech mehr als 2 Punkte Essenz verloren hat, kann diesen Weg beschreiten. Dies erlaubt ihm, bei seiner Initiation \ref{sec:Initiation} stattdessen den Effekt einer Cyberinitiation \ref{sec:Cyberinitiation} zu erhalten. Der Adept erhält jedoch nur ein Essenzloch der Größe $0.5 \text{Essenz}$ statt der für Mundane üblichen $1.0 \text{Essenz}$. Für seinen maximalen Cyberinitiatengrad gilt jedoch die selbe Einschränkung wie bei der normalen Cyberinitiation: Der maximale Cyberinitiatengrad ist immer gleich dem (aufgerundeten) halben Essenzverlust eines Charakters. \textbf{Wichtiger Hinweis}: Wählt ein Charakter bei seiner Initiation diese Option, steigen sowohl sein Initiatengrad, als auch sein Cyberinitiatengrad um 1.}

\subsection{Magieradepten}
\homebrew{Magieradepten funktionieren hier wie in Shadowrun 4. Sie haben zwei separate Magieattribute: Eines für die Adeptenseite und eines für die Zaubererseite. Wann immer ein Magieradept sein Magieattribut erhöht, muss er sich entscheiden, ob er diesen zusätzlichen Punkt Magie in die eine oder die andere Seite investiert. Die einzelnen Seiten funktionieren dann jedoch wie ein reiner Adept bzw wie ein reiner Zauberer. Ein Magieradept mit Adept 2 und Zauberer 4 erhält bspw 2KP von der Adeptenseite, während abb einer Kraftstufe von 5 der Entzug seiner Zauber physisch wird.}\\
\homebrew{Außerdem kann ein Magieradept keine Vorteile für Adepten/Zauberer erwerben.}
