\section{Magie}

\subsection{Hausregeln}
\begin{itemize}
    \item Beim Entzugswiderstand werden immer nur die natürlichen Attribute des widerstehenden Charakters verwendet. Technologische und/oder magische Attributsverstärkungen werden dabei also nicht angerechnet.
    \item Reagenzien können nicht verwendet werden, um Limits zu hacken.
    \item Ein Charakter kann von jeder Fokusart immer nur einen gleichzeitig aktiv haben. Beispielsweise also einen Zauberspeicher (insgesamt; nicht pro Zauberart), einen Herbeirufen-Fokus, einen Metamagie-Fokus usw.
    \item Bei Proben sind Würfelpool-Boni durch Foki auf $4$ begrenzt.
    \item Foki haben andere Verfügbarkeiten. Kraftfoki haben $6\text{E}/\text{Kraftstufe}$, Waffenfoki haben $5\text{E}/\text{Kraftstufe}$, alle anderen haben $4\text{E}/\text{Kraftstufe}$.
    \item Der maximale Initiatengrad eines Charakters ist $\frac{\text{Magie}}{2} \text{(aufgerundet)}$.
\end{itemize}

\subsection{Erweiterte magische Fertigkeiten}
Die Fertigkeiten \textbf{Fokusherstellung} und \textbf{Entzaubern} haben zusätzliche Anwendungsgebiete. Genauer gesagt werden im Verlauf der Kampagne verzauberte Objekte auftauchen, welche keine Foki sind. Um mit diesen zu interagieren, werden diese beiden Fertigkeiten benötigt.

\subsection{Arten der Magie}
\paragraph{Magieradepten}
Für Magieradepten kosten Kraftpunkte je $10\text{ Karma}$, statt wie RAW 5.\\

\subsubsection{Aspektzauberer}
Aspektzauberer haben zwar den Nachteil der Eingeschränktheit in der Wahl ihrer Mittel. Dafür bekommen sie hier jedoch einige Boni - je nach Spezialgebiet. Ein Aspektzauberer erhält bei Erwerb einer Spezialisierung/Expertise auf eine dieser Fertigkeiten den entsprechenden Bonus.
\paragraph{Aspektbeschwörer}
\begin{itemize}
    \item[Berschwören] Der Charakter hat einen zusätzlichen Herbeirufen-Würfel.
    \item[Binden] Der Charakter hat einen zusätzlichen Binden-Würfel.
    \item[Verbannen] ?
\end{itemize}

\paragraph{Aspektverzauberer}
\begin{itemize}
    \item[Alchemie] Alchemische Erzeugnisse haben eine um $1$ höhere Wirksamkeit.
    \item[Fokusherstellung] ?
    \item[Entzaubern] ?
\end{itemize}

\paragraph{Aspekthexer}
\begin{itemize}
    \item[Spruchzauberei] Der Charakter kann einen zusätzlichen Zauber ohne Würfelpoolmalus aufrechterhalten. 
    \item[Ritualzauberei] Der Charakter
    \item[Antimagie] Der Charakter
\end{itemize}

\subsection{Magische Vor-/Nachteile}
\paragraph{Astralsucht}
Ein Charakter mit diesem Vor-/Nachteil kann astral wahrnehmen, OHNE den Würfelpoolmalus von $-2$ auf alle physischen Proben zu erhalten. Dafür erhält er einen Malus von $-2$ auf all seine Proben, wenn er NICHT astral wahrnimmt.


\subsection{Geister}
Hier eine Übersicht der verwendeten Hausregeln:
\begin{itemize}
    \item Beim Herbeirufen wir die vergleichende Probe gegen die DOPPELTE Kraftstufe des Geistes abgelegt. Der Entzug ist dann gleich der Anzahl der Erfolge des Geistes statt der doppelten Anzahl der Erfolge.
    \item Beim verstärkten Panzer materialisierter Geister fällt der Schwellenwert weg.
    \item Beschwörer können mittels des Systems für Astrale Reputation (in \autoref{sec:Astrale Reputation}) weitere Effekte für ihre beschworenen Geister erzielen.
    \item Ein Charakter kann von jeder Geisterart seiner Tradition nur jeweils einen gleichzeitig gebunden haben. Solange ein Geist einer Art gebunden ist, kann der Charakter diese Geisterart nicht mehr herbeirufen.
    \item Auch gebundene Geister sind (teilweise) an den Tag-/Nachtzyklus der Welt gekoppelt. Sie verschwinden bei jedem Sonnenauf- und -untergang für $\text{Kraftstufe}$ Kampfrunden zurück auf ihre Metaebene. Dadurch enden beispielsweise durch den Geist aufrechterhaltene Zauber. Ist der gebundene Geist auf einer Mission, kehrt er jedoch selbstständig an den Ort, an dem er verschwand zurück und führt seinen Auftrag fort.
\end{itemize}

\subsubsection{Traditionen und Geister}
Jeder Charakter hat ein eigenes Verhältnis zur Geisterwelt.\\
Anhänger schamanischer (charismabasierter) Traditionen sehen Geister als gleichberechtigte Partner und teilweise sogar Freunde. Sie rufen gerne immer wieder die gleiche Entität herbei, wenn sie einen Geist einer Art rufen. Sie bemühen sich um eine gute astrale Reputation. Daher setzen sie das Binden nur sehr sparsam ein und vereinbaren in der Regel Kompensation (Erfüllen einer Quest, Übergabe von Karma an den Geist, etc.), sollten sie doch einmal einen Geist binden.\\
Hermetische (logikbasierte) Traditionen hingegen halten Geister für reine Werkzeuge, welche ihnen beim Erreichen ihrer Ziele helfen sollen. Sie würden niemals einen Geist nach seinem Namen fragen und setzen rücksichtslos das Binden ein, um sich Geister gefügig zu machen. Ihre astrale Reputation ist traditionell schlecht.\\
Chaotische (intuitionsbasierte) Traditionen haben ein ambivalentes Verhältnis zur Geisterwelt. Einerseits betrachten sie Geister für auf ihrer eigenen Stufe stehend, jedoch nicht wie (potentielle) Freunde, sondern eher wie Geschäftspartner, mit denen man hart verhandelt und die man zu übervorteilen versucht, aber die man auch mal zum Essen einlädt. Diese Traditionen haben stark schwankenden astrale Reputationen.

\subsubsection{Herbeirufen}
Beim Herbeirufen wird die vergleichende Probe gegen die DOPPELTE Kraftstufe des Geistes abgelegt. Der Entzug ist dann gleich der Anzahl der Erfolge des Geistes statt der doppelten Anzahl der Erfolge.\\
Bei einem Gleichstand bei der vergleichenden Probe erscheint der Geist dennoch. Dann werden ggf. Herbeirufen-Würfel geworfen. Gewährt einer dieser Würfel dem Beschwörer Dienste, so ist der Geist ihm diese schuldig und bleibt. Ist dies nicht der Fall, verhält sich der Geist entsprechend der astralen Reputation des Beschwörers. 
\subsubsection{Astrale Reputation}
Die astrale Reputation eines Charakters kann einen Wert von $-10$ bis $10$ annehmen. Diese hat folgende Auswirkungen:
\begin{itemize}
    \item $\geq +9$: 3 Herbeirufen-Würfel
    \item $\geq +6$: 2 Herbeirufen-Würfel.
    \item $\geq +3$: 1 Herbeirufen-Würfel.
    \item $\geq -3$: Ein Geist greift einen einmal an, sollte er die Gelegenheit dazu erhalten.
    \item $\geq -6$: Ein Geist greift einen 2-3 mal an, sollte er die Gelegenheit dazu erhalten. 
    \item $\geq -9$: Ein Geist versucht einen zu töten, sollte er die Gelegenheit dazu erhalten. 
\end{itemize}
Ein Charakter hat bei jeder seiner Tradition zugänglichen Geisterart eine gesonderte astrale Reputation. Diese ist abhängig von seinem Verhalten gegenüber Geistern dieser Art.\\
Beispiele für negative Aktionen gegenüber Geistern sind Binden, verdrängen lassen, der Dienst \textit{Zauber binden}, das Beauftragen mit einer Aufgabe, welche nicht der Tradition entspricht etc.\\
Auch Geister, welche von anderen herbeigerufen wurden, werden bei einem negativen Geisterruf versuchen, einem zu schaden - sofern ihre Befehle ihnen dies erlauben.

\subsubsection{Herbeirufen}
Ruft ein Charakter einen Geist herbei, so werden je nach astraler Reputation Herbeirufen-Würfel geworfen. Dieser hat - je nach Würfelergebnissen - folgende Auswirkungen:
\begin{enumerate}
    \setcounter{enumi}{3}
    \item[1,2] $\text{Entzug}-1$
    \item[3,4] $\text{Dienste}+1$
    \item[5,6] $\text{Kraftstufe}+1$
\end{enumerate}
\subsubsection{Binden}
Will ein Charakter einen Geist binden, so muss dies direkt nach dem Herbeirufen initiiert werden. Bindet ein Charakter einen Geist, gehen alle Boni durch etwaige Herbeirufen-Würfel verloren. Außerdem kann ein Charakter keine Herbeirufen-Würfel verwenden, solange ein Geist an ihn gebunden ist.\\
Jedes Binden senkt die astrale Reputation des Charakters bei der herbeigerufenen Geisterart um 1.
