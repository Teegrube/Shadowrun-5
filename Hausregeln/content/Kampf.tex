\section{Kampf}
\label{sec:kampf}

Kämpfe in Shadowrun sind rundenbasiert. Sie sind daher in \textbf{Kampfrunden} unterteilt, welche jeweils 3 Sekunden eines Kampfes abbilden.
Eine Kampfrunde wird dabei in mehreren \textbf{Initiativedurchgängen} abgehandelt.
In jeden Iniativedurchgang handeln die Charaktere nacheinander, wobei die Reihenfolge durch ihr \textbf{Initiativeergebnis} festgelegt wird.
Ist ein Charakter aufgrund seines Initiativeergebnisses an der Reihe, hat er eine \textbf{Handlungsphase} zur Verfügung, in welcher er agieren kann.
Pro Initiativedurchgang kann er eine \textbf{Freie Handlung} ausführen, welche er auch außerhalb seiner Handlungsphase zu einem beliebigen Zeitpunkt ausführen kann.
Innerhalb seiner Handlungsphase kann er außerdem entweder eine \textbf{Komplexe Handlung} oder zwei \textbf{Einfache Handlung}en durchführen, wobei jede Einfache Handlung auch dazu verwendet werden kann, stattdessen eine Freie Handlung auszuführen.
Außerdem können die Charaktere sich in jeder 
\subsection{Ablauf einer Kampfrunde}
\label{sec:kampfrunde}
Eine Kampfrunde läuft in mehreren Schritten ab:
\paragraph{1. Initiativeprobe}
Jeder Charakter wirft seine Initiativewürfel. Die geworfene Augenzahl wird auf seinen Initiativewert addiert. Das Ergebnis bildet dann das \textbf{Initiativeergebnis}, welches von der Spielleiterin in absteigender Reihenfolge notiert wird, um so die \textbf{Initiativereihenfolge} festzulegen.
\paragraph{2. Initiativedurchgang}
In der zuvor festgelegten Reihenfolge der Initiativeergebnisse führen nun alle Charaktere ihre \textbf{Handlungsphasen} durch. 
\paragraph{3. Handlungsphase}
Zu Beginn seiner Handlungsphase kann ein Charakter sich dazu entscheiden, eine \textbf{Verzögerte Handlung} anzusagen, um seine Handlungsphase später durchzuführen. Tut er dies, endet seine Handlungsphase sofort und der nächste Charakter ist an der Reihe.
Dafür darf der seine Handlung verzögernde Charakter ab sofort vor oder nach der Handlungsphase eines beliebigen anderen Charakters entscheiden, sofort eine Handlungsphase durchzuführen.\\\\
In seiner Handlungsphase darf ein Charakter dann eine \textbf{Komplexe Handlung} oder zwei \textbf{Einfache Handlung}en durchführen. Außerdem darf er ohne dafür eine Handlung aufzuwenden beliebig \textbf{Bewegung}en durchführen. \textbf{Wichtig}: Beachte dazu auch, dass sich die maximale Bewegungsreichweite eines Charakters \textbf{nicht} zu Beginn jeder Handlungsphase wieder auffüllt, sondern \textbf{nur zu Beginn jeder Kampfrunde}!

