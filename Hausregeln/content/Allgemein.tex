\section{Allgemein}
\label{sec:Allgemein}
\subsection{Vorteile}
\paragraph{HB: Erhöhte Konzentrationsfähigkeit}
Dieser Vorteil ist gebannt.
\paragraph{HB: Weg des Ausgebrannten}
Siehe \ref{qual:Ausgebrannt}.



\subsection{Edge}
\paragraph{Patzer ausbügeln}
Das Ausbügeln eines Patzers muss angesagt werden, \textit{bevor} über die Auswirkungen des Patzers entschieden wurde. 

\paragraph{Edge verheizen}
Wurde ein Punkt Edge verheizt, so kostet der Rückkauf des verheizten Punkts Edge \textbf{immer} $15 \text{Karma}$. Außerdem kann kein weiteres Edge verheizt werden, bis dieser Punkt zurückgekauft wurde.

\subsection{Zweiter Versuch}
Die Regeln für einen \textit{zweiten Versuch} (GRW S.51) gelten nicht, es sei denn, die Spielleitung stimmt zu. Um einen zweiten Versuch durchführen zu können, muss der Charakter zuvor einer anderen Tätigkeit nachgegangen sein, um den Kopf freizukriegen. Dies könnte beispielsweise (je nach Probe) ein Spaziergang, ein Arbeitstag oder ein Treffen mit einer Freundin sein. Teilweise muss dafür auch ein ganzer Tag (oder noch länger) verstrichen sein. Hat sich an den Modalitäten der Probe etwas signifikant geändert (bspw. dadurch, dass der Charakter nun eine Karte für seine Probe zur Orientierung aufgetrieben hat), so kann der Malus ggf zurückgesetzt werden.