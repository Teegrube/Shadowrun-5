\section{Bodytech}
\label{sec:Bodytech}

\subsection{Essenzlöcher}
(BS: S.87, Kasten)\\
Wird ein Stück Bodytech ausgebaut, entsteht ein \textbf{Essenzloch} mit einer Größe der Essenzkosten des ausgebauten Teils. Dieses kann dann mit anderer Bodytech gefüllt werden, bevor die restliche Essenz des Charakters angetastet werden muss.\\
Baut man also bspw ein Teil, welches $0.5$ Essenz kostete, aus, und dann ein Teil, welches $0.9$ Essenz kostet ein, werden $0.5$ der Essenzkosten des neuen Teils durch das Essenzloch bezahlt.\\

\subsection{Upgrades}
(BS: S.87, Kasten)\\
Bodytech kann geupgraded werden. RAW wird dabei die Alte komplett durch Neue ersetzt. \homebrew{Hier wird jedoch tatsächlich ein Upgrade durchgeführt. Dabei kann auch die Gütekategorie schon bestehender Bodytech verändert werden. Eine Ausnahme bildet jedoch Kultivierte Bioware, welche zwar in der Stufe, nicht aber in der Gütekategorie geupgraded werden kann.}
\subsection{HB: Cyberinitiation}
\label{sec:Cyberinitiation}
\homebrew{Die Kosten der Cyberinitiation sind genau die selben wie die der normalen Initiation \ref{sec:Initiation}. Allerdings gewährt diese keine Boni auf Magie, sondern ein Essenzloch der Größe $1.0 \text{Essenz}$, welches mit Bodytech gefüllt werden kann. Der maximale Cyberinitiatengrad ist immer gleich dem (gerundeten) halben Essenzverlust eines Charakters. Ein Charakter, welcher beispielsweise bereits $4.5 \text{Essenz}$ verbraucht hat kann also bis zu zwei Cyberinitiatengrade erhalten.}\\
\homebrew{Charaktere mit einem Magie- oder Resonanzattribut können keine Cyberinitiatengrade erwerben. Die einzige Ausnahme sind Adepten, welche dem \textbf{Weg des Ausgebrannten} \ref{qual:Ausgebrannt} (SG, S.207f)}